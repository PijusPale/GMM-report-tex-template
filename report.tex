\documentclass[conference]{IEEEtran}
\IEEEoverridecommandlockouts

\usepackage[utf8]{inputenc}
\usepackage[lithuanian]{babel}
\usepackage[T1]{fontenc}


% The preceding line is only needed to identify funding in the first footnote. If that is unneeded, please comment it out.
\usepackage{cite}
\usepackage{amsmath,amssymb,amsfonts}
\usepackage{algorithmic}
\usepackage{graphicx}
\usepackage{textcomp}
\usepackage{xcolor}
\def\BibTeX{{\rm B\kern-.05em{\sc i\kern-.025em b}\kern-.08em
    T\kern-.1667em\lower.7ex\hbox{E}\kern-.125emX}}

\usepackage[labelsep=endash]{caption}
\renewcommand{\figurename}{pav}
\renewcommand\IEEEkeywordsname{Raktažodžiai}

\usepackage{lipsum}  




\begin{document}

\title{Matematikos Sprendimas}

\author{\IEEEauthorblockN{Dominykas Dulevičius}
\IEEEauthorblockA{\textit{Informatikos institutas} \\
\textit{Matematikos ir informatikos fakultetas}\\
Vilnius, Lietuva \\
dominykas.dulevicius@mif.stud.vu.lt}
\and
\IEEEauthorblockN{Arnas Bulka}
\IEEEauthorblockA{\textit{Informatikos institutas} \\
\textit{Matematikos ir informatikos fakultetas}\\
Vilnius, Lietuva \\
arnas.bulka@mif.stud.vu.lt}
\and
\IEEEauthorblockN{Pijus Kizerskis}
\IEEEauthorblockA{\textit{Informatikos institutas} \\
\textit{Matematikos ir informatikos fakultetas}\\
Vilnius, Lietuva \\
pijus.kizerskis@mif.stud.vu.lt}
}

\maketitle

\begin{abstract}
Projektinio darbo metu tyrėme irlyginome internete paplitusių populiarių modelių
galimybes ir tikslumą sprendžiant matematinius uždavinius ir lyginome rezultatus su
TP-Transformer tipo modeliu, kuris buvo specialiai aptreniruotas su DeepMind kūrėjų
„Mathematics-Dataset“ duomenų rinkiniu.
\end{abstract}

\begin{IEEEkeywords}
TP - Tenzoriaus-Produkto
\end{IEEEkeywords}

\section{Įvadas}
Pristatote problemą

\section{Duomenų Rinkinys}

\subsection{Metodas A}

Pristatote naudojamus metodus

\begin{equation}
\mathcal{L}({\bf X}, {\bf Y}) = \frac{1}{w \cdot h} \sum_{i=1}^h \sum_{j=1}^w (X_{i, j} - Y_{i, j})^2
\label{eq:lygtis1}
\end{equation}

Taikyta nuostolių funkcija ~\eqref{eq:lygtis1}.

\begin{align}
y & = f(x) \nonumber \\
f & = f_1(f_2(x))
\label{eq:lygtis2}
\end{align}

Taikytas modelis ~\eqref{eq:lygtis2}.


\section{Transformer ir TP-Transformer}
Aprašote naudojamus duomenis.

\subsection{Equations}



\subsection{Paveikslėliai ir lentelės}

Paveikslėlį cituojame ``~\ref{fig} pav.''.

Paveikslėlį cituojame ``~\ref{tab1} lentelė''.

\begin{table}[htbp]
\caption{Lentelės aprašas}
\begin{center}
\begin{tabular}{|c|c|c|c|}
\hline
a & b & c &  d \\
\hline
\end{tabular}
\label{tab1}
\end{center}
\end{table}

% \begin{figure}[!h] % įterpti čia
% \centerline{\includegraphics{fig1.png}}
% \caption{Paveikslėlio aprašas.}
% \label{fig}
% \end{figure}


\section{Naudojami Apmokyti Modeliai}
Lyginti su TP-Transformer pasirinkome 5 iš anksto apmokytus klausimų-atsakymų modelius: \textbf{
iAsk.Ai}, .... Šių modelių tikslumą matematikos uždavinių sprendimui lyginsime su \textbf{ischlag/TP-Transformer} modeliu,
specialiai pritaikytų matematikos uždaviniams spręsti.

\subsection{ischlag/TP-Transformer}
Šio modelio sukūrimui į standartinį transformerio tipo modelį buvo inkorporuota Tenzoriaus-Produkto
reprezentacija, tokiu būdu pagerinant išskirtinius sąryšius tarp transformerio vidinių struktūrų. Modelis buvo apmokytas
naudojant „Mathematics-Dataset“, kuris buvo išskaidytas į 56 skirtingas matematikos šakas, siekiant
pagerinti modelio tikslumą bei atpažinamumą

\subsection{iAsk.Ai}
Šis kalbos variklis naudoja panašias technologijas, kaip ir \textbf{ChatGPT} variklis, tačiau
papildomas dėmesys yra skiriamas optimizuoti natūralios kalbos aprodorojimo modelį. iAsk AI taip pat susideda iš specialiai pritaikyto,
didelio masto Transformerio kalbos modelio. Šis modelis buvo išskirtinai mokomas remiantis
patikimiausiais ir autoritetiškiausiais literatūros bei interneto šaltiniais,
kas suteikia iAsk AI galimybę atsakyti į klausimus objektyviai,
faktiškai ir be potencialaus subjektyvumo, kurio galėtų būti ChatGPT.


\section{Tyrimo Aprašymas}

\subsection{Eiga}
Iš „Mathematics-Dataset“ duomenų rinkinio 56 kategorijų
\section{Rezultatai}

\section{Išvados}


% Cituojame šaltinį \cite{lecun2015deep}.

\bibliographystyle{plain}
\bibliography{saltiniai}

\end{document}
